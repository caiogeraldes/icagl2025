\documentclass[article,a4paper,12pt]{memoir}

\usepackage{fontspec}
\usepackage{csquotes}
\usepackage[main=english]{babel}
\setmainfont{Brill}
\setmonofont[Scale=0.8]{Noto Sans Mono}
\usepackage{hyperref}

\usepackage{linguex}

\usepackage[
  backend=biber,
  uniquename=init,
  giveninits,
  style=authoryear
  ]{biblatex}
\addbibresource{./biblio.bib}


\title{Case attraction on Ancient Greek Infinitive Clauses}
\author{Caio B. A. Geraldes}

\begin{document}

\begin{center}
  {\large\textbf{\thetitle}}\\
  \bigskip
  \theauthor\\
  \bigskip
\noindent\textbf{Key-words:} syntax; case attraction; quantitative methods
\end{center}
  
Case attraction on infinitive clauses (contrast~\autoref{gloss:noattr}
with~\autoref{gloss:attr} below) has
been analysed over the recent years as a idiosyncratic and strictly
syntactic construction (\cite{Tantalou2003},~\cite{Spyropoulos2005} and 
\cites{Sevdali2013b}{Sevdali2013}), lacking many parallels across either 
ancient or modern languages and across other linguistic \emph{phenomena}. 
Previously (Geraldes 2020), I have tried to show that case attraction is 
similar in nature and distribution to \emph{agreement / concord}, namely 
similar to the cases of \emph{non-canonical agreement} and \emph{long 
distance agreement} (as discussed in~\cite{Corbett2006}), due to its 
association with semantic and pragmatic features, using a \emph{corpus} 
comprised of Herodotus, Plato and Xenophon. Effects of authorship, genre and 
dialect were also identified, with philosophical dialogues and attic presenting
a higher likelihood to display case attraction. The issues with the
evidence and analysis made available was the sample size, which limited
what conclusions could be drawn from the data collected, and the sole
use of single variate methods to estimate the effects and association
between variables and the occurrence of attraction, which made it
impossible to untangle structural linguistic effects from
extra-linguistic factors.


\ex.\ag.\label{gloss:noattr}συμβουλεύει {τῷ Ξενοφῶντι} ἐλθόντα {εἰς Δελφοὺς} ἀνακοινοῶσαι {τῷ θεῷ} {περὶ τῆς πορείας}.\\
  advice.3SG X.DAT.SG going.ACC.SG to-Delphi ask.INF the-god.DAT.SG
  about-the-travel\\
  He advices Xenophon to go to Delphi and ask the god about the travel.
  (Xen. Anab. 3 1 5)
\bg.\label{gloss:attr}ἀφῆκε μοι ἐλθόντι {πρὸς ὑμᾶς} λέγειν τἀληθῆ.\\
  allowed.3SG PRON.1SG.DAT going.DAT.SG in-front-of-you say.INF
  the-truth.ACC.\\
  He allowed me to go and speak the truth in front of you. (Xen. Hell. 6
  1 13)


Expanding on the aforementioned work, this aims to further the analysis
of case attraction of infinitival predicates of Ancient Greek,
specifically on attic and ionic literary sources. The new \emph{corpus}
includes oratory, dramatic and historiographic literature from Classical
Athens and a revised version of the corpus of sentences drawn from
Herodotus, Plato and Xenophon (published as Geraldes 2021).

The nucleus of this work lies on a new quantitative analysis of the data
built as to adequately estimate the direct effects of linguistic factors
(class of main verb and infinitival verb, constituent distance, and
part-of-speech of the predicates) on the occurrence of case attraction
accounting also for the interaction between such factors and
extralinguistic factors such as authorship and genre. This analysis
relies on the graphical causal analysis (e.g.~\cite{Pearl2009})
and Bayesian modeling (e.g.~\cite{McElreath2020}), which I argue could 
enhance the results of data driven linguistic research by including at the 
quantitative analysis the qualitative knowledge built on the Ancient Greek 
and general linguistics. This is particularly important as semantic and
pragmatic features often can only be represented in the data by proxies
(word classes, constituent distance and word order).

Furthermore, the evidence gathered reinforces the conclusion that case
attraction was strongly conditioned by the semantic and pragmatic
features of the sentence.

\printbibliography%

\end{document}
