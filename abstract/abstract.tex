\documentclass[article,a4paper,12pt]{memoir}

\usepackage{fontspec}
\usepackage{csquotes}
\usepackage[main=english]{babel}
\setmainfont{Brill}
\setmonofont[Scale=0.8]{Noto Sans Mono}
\usepackage{hyperref}

\usepackage{linguex}

\usepackage[
  backend=biber,
  uniquename=init,
  giveninits,
  style=authoryear
  ]{biblatex}
\addbibresource{./biblio.bib}


\title{Rethinking case attraction on Ancient Greek infinitive clauses}
\author{Caio B. A. Geraldes}

\begin{document}

\begin{center}
  {\large\textbf{\thetitle}}\\
  \bigskip
  \theauthor\\
  \bigskip
\noindent\textbf{Key-words:} syntax; case attraction; quantitative methods
\end{center}
  
Case attraction on infinitive clauses (contrast~\autoref{gloss:noattr}
with~\autoref{gloss:attr} below) has
been analysed over the recent years as a idiosyncratic and strictly
syntactic construction (\cite{Tantalou2003},~\cite{Spyropoulos2005} and 
\cites{Sevdali2013b}{Sevdali2013}), lacking many parallels across either 
ancient or modern languages and across other linguistic \emph{phenomena}. 
In this paper, I offer a new assessment of the process, arguing that case
attraction is similar in nature and distribution to \emph{agreement / concord},
namely similar to the cases of \emph{non-canonical agreement} and \emph{long 
distance agreement} (as discussed in~\cite{Corbett2006}).

\ex.
\ag.\label{gloss:noattr}συμβουλεύει {τῷ Ξενοφῶντι} ἐλθόντα {εἰς Δελφοὺς}
    ἀνακοινῶσαι {τῷ θεῷ} {περὶ τῆς πορείας}.\\
    advice.3SG X.DAT.SG going.ACC.SG to-Delphi ask.INF the-god.DAT.SG
    about-the-travel\\
    He advices Xenophon to go to Delphi and ask the god about the travel.
    (Xen. Anab. 3.1.5)
\bg.\label{gloss:attr}ἀφῆκε μοι ἐλθόντι {πρὸς ὑμᾶς} λέγειν τἀληθῆ.\\
    allowed.3SG PRON.1SG.DAT going.DAT.SG in-front-of-you say.INF
    the-truth.ACC.\\
    He allowed me to go and speak the truth in front of you. 
    (Xen.~Hell.~6.1.13)

The evidence across natural languages shows that \emph{non-canonical} agreement
\slash{} concord takes place in a non-deterministic fashion as semantic or
pragmatic features of the sentence appear more marked, which is similar to the
contexts associated with case attraction assumed by grammarians as early as
\textcite{Buttmann1826}, contexts in which some sort of \emph{emphasis} is
assigned to the target of attraction makes it more likely to be attracted.
Although the intuition seems to be sound, there is little to no specificity in 
what is denoted by \emph{emphasis} and the explanation is prone to \emph{ad hoc}
interpretations.

Using data from literary sources of Classical Greek, including oratory speeches,
drama, historiography and philosophical dialogues from Attic and Jonic sources,
I provide a data driven quantitative analysis of the contexts in which case
attraction is a possible agreement \slash{} concord resolution. The addition of
Jonic sources is due to the fact that it has been assumed that case attraction is
more common if not the rule in the Attic dialect
(e.g.~\cite[\emph{passim.}]{Buttmann1826} and~\cite[\emph{ad loc.}]{Cooper1997}).
The data has been collected and annotated in a combination of manual and
computational methods using the Diorisis Ancient Greek
Corpus~\parencite{TheDiorisisAncientGreekCorpus}.

A quantitative analysis of case attraction requires caution in the
methodological approach, as semantic and pragmatic features are often latent,
i.e.\ not explicitly present in the morpho-phonological level, and the use of
proxies such as word classes, constituent distance and word order may hinder the
quality of the results and the causal inference built upon them.
As such, our analysis will rely on the causality analysis (e.g.~\cite{Pearl2009})
and Bayesian modeling (e.g.~\cite{McElreath2020}), which I argue could 
enhance the results of data driven linguistic research by including at the 
quantitative analysis the qualitative knowledge built on the Ancient Greek 
and general linguistics.
The analysis must be thus twofold. Firstly, it assess how the linguistic and
extralinguistic factors could reasonably be causally linked with case
attraction, so as to inform the statistical modeling and tell what effects are
possible to estimate from the data.
Later, a general linear model is build as to adequately estimate the direct
effects of semantic and pragmatic factors on case attraction and the
interactions between linguistic and extralinguistic variables.

\printbibliography%

\end{document}
